\documentclass[12pt]{article}

\author{Corbin T. Rochelle}
\title{QuC HW04}
\date{\today}

\usepackage{physics}

\begin{document}
\maketitle

\section*{Question 1}
\subsection*{A.}
The initial realiziation I made was that that the Eigenvectors, $\ket{E_1}$, $\ket{E_2}$, $\ket{E_3}$, and $\ket{E_4}$, were the same values for all values of \texttt{ham2factorC};
however the Eigenvalues change with a different pattern: namely $\ket{E_1}$ is the negation of $\ket{E_2}$ and $\ket{E_3}$ is the negation of $\ket{E_4}$, while $\ket{E_2}$ is calculated with $1+\texttt{ham2factorC}$ and $\ket{E_4}$ is calculated with $1-\texttt{ham2factorC}$!
ASK ABOUT THE FINAL PART OF THIS QUESTION
\subsection*{B.}
The first thing I notice is that the Eigenvalues follow the same pattern described for the system in question 1A. 
The second thing I notice is that although the Eigenvalues follow that pattern, the Eigenvectors do not. 
The $\ket{\psi_{initial}}$ is the same as $\ket{\psi_{final}}$ because the $P_j(t)$ function requires the norm to be taken of both wave functions. 
The norm of $\ket{\psi_{initial}}=\ket{\psi_{final}}$ for all four values, creating a straight line.
The values in the vector are changing, but the norm of the values are always the same. 
\subsection*{C.}


\section*{Question 2}
\subsection*{A.}
\subsection*{B.}
\subsection*{C.}
\subsection*{D.}

\end{document}