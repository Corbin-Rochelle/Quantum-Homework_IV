\documentclass[12pt]{article}

\author{Corbin T. Rochelle}
\title{QuC HW04}
\date{\today}

\usepackage{physics}
\usepackage{amsmath}

\begin{document}
\maketitle

\section*{Question 1}
\subsection*{A.}
The initial realiziation I made was that that the Eigenvectors, $\ket{E_1}$, $\ket{E_2}$, $\ket{E_3}$, and $\ket{E_4}$, were the same values for all values of \texttt{ham2factorC};
however the Eigenvalues change with a different pattern: namely $\ket{E_1}$ is the negation of $\ket{E_2}$ and $\ket{E_3}$ is the negation of $\ket{E_4}$, while $\ket{E_2}$ is calculated with $1+\texttt{ham2factorC}$ and $\ket{E_4}$ is calculated with $1-\texttt{ham2factorC}$!
We can explain $P_j(t)$ easily by looking at the resulting $\ket{\psi_{final}}$!
For different values of \texttt{ham2factorC}'s, we see different values of $\ket{\psi_{final}}$, but if we take the norm of all of the resulting states, we get $\frac{1}{2}$. 
This means when we take $\lvert\bra{j}\ket{\psi_{final}}\rvert^2$, the $\bra{j}$ will select one value from $\ket{\psi_{final}}$ and when taking the square of the $\frac{1}{2}$, we get $\frac{1}{4}$, which is the values we see for all $P_j(t)$.

\subsection*{B.}
The first thing I notice is that the Eigenvalues follow the same pattern described for the system in question 1A. 
The second thing I notice is that although the Eigenvalues follow that pattern, the Eigenvectors do not. 
The $\ket{\psi_{initial}}$ is the same as $\ket{\psi_{final}}$ because the $P_j(t)$ function requires the norm to be taken of both wave functions. 
The norm of $\ket{\psi_{initial}}=\ket{\psi_{final}}$ for all four values, creating a straight line.
For all values of t, the vectors are changing, but the norm of the values are always the same. 
This is because $H_{total}$ and $\ket{\psi_{final}}$ are orthogonal to each other! 

\subsection*{C.}
I found two Hamiltonians that meet the following criteria:
\begin{center}
    $H_{total} = (\sigma^x \bigotimes \sigma^x) + C (\sigma^x \bigotimes \sigma^x)$\\
    and\\
    $H_{total} = (\sigma^y \bigotimes \sigma^y) + C (\sigma^y \bigotimes \sigma^y)$\\
    and\\
    $H_{total} = (\sigma^z \bigotimes \sigma^z) + C (\sigma^z \bigotimes \sigma^z)$
\end{center}
For all of these, the bell state is $\ket{\psi_{initial}}$=
$
\begin{pmatrix} 
    \frac{1}{\sqrt{2}} \\ 0 \\ 0 \\ -\frac{1}{\sqrt{2}}
\end{pmatrix}
$
and $\ket{\psi_{final}}=$
$
\begin{pmatrix} 
    x \\ 0 \\ 0 \\ x
\end{pmatrix}
$, where the norm of $x=\frac{1}{\sqrt{2}}$.
From this we can see two of the $j$ Cbits will select one of the x's, namely the $\ket{00}$ or $\ket{11}$, which have the $P_j(t)=0.5$, while the other $j$s will multiply by the zeros and have $P_j(t)=0$.
We know $\ket{\psi_{final}}$ will always have two 0s because the eigenvectors are always orthogonal to $\ket{\psi_{final}}$.

\section*{Question 2}
\subsection*{A.}
 For any system of $N$ qbits, $E_{GS}=-N$, while $E_2=-(N-2)$.
 And I have checked and verified up to 6 qbits. 
\subsection*{B.}
See my Mathematica file!
\subsection*{C.}
See my Mathematica file!
\subsection*{D.}
See my Mathematica file!

\end{document}